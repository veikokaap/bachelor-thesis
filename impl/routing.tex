%!TEX root = ../thesis.tex
\documentclass[..thesis.tex]{subfiles}

\begin{document}

Most of the commands sent by the debuggers are mostly stateless and don't manipulate or change the state of the JVM.
Such commands usually ask for some information and receive it in the reply without the JVM having to alter its behaviour.
No special handling of such commands is needed apart from changing the packet id due to reasons discussed in section \ref{sec:impl_id}.

There are however commands which either control the execution of the program in JVM or request for notifications about certain events in the future.
Some of the most widely used commands which alter the execution of the debugged program are resume and suspend commands.
These tell the JVM to suspend or resume all execution in a single thread or for all threads.
The proxy server must have special logic to handle these commands since sending these commands directly to the JVM can cause unpredictable behaviour.

For example, if two debuggers send a suspend command to the proxy server, then the proxy server needs to remember that both debuggers want the JVM to be suspended, but won't suspend the JVM twice.
When one of these debuggers later sends a resume command to the proxy server, then it won't be sent on to the JVM because one debugger still needs the JVM to stay suspended.
The resume command is only sent on when both debuggers have sent it.
Otherwise, the second debugger would think the JVM is still suspended, but actually it has been resumed by the first one.

\subfile{./graphs/suspend-proxy-2.tex}

As can be seen on the above graph, the first suspend command is forwarded to the JVM, but the second one isn't. 
With resume command it's the opposite: the first resume command is not forwarded, but the second one is.
It's also important to mention that even though the command is not always forwarded to the JVM, the proxy server still has to reply to the debugger since each command packet sent from a debugger expects to receive a reply packet from the JVM.

\end{document}
