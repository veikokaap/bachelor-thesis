%!TEX root = ../thesis.tex
\documentclass[..thesis.tex]{subfiles}

\begin{document}

Most of the commands sent by the debuggers are mostly stateless and don't manipulate or change the state of the JVM.
Such commands usually ask for some information and receive it in the reply without the JVM having to alter its behaviour.
No special handling of such commands is needed apart from changing the packet id due to reasons discussed in section \ref{sec:impl_id}.

There are however commands which either control the execution of the program in JVM or request for notifications about certain events in the future.
Some of the most widely used commands which alter the execution of the debugged program are resume and suspend commands.
These tell the JVM to suspend or resume all execution in a single thread or for all threads.
The proxy server must have special logic to handle these commands since sending these commands directly to the JVM can cause unpredictable behaviour.

For example, if two debuggers send a suspend command to the proxy server, then the proxy server needs to remember that both debuggers want the JVM to be suspended, but won't suspend the JVM twice.
When one of these debuggers later sends a resume command to the proxy server, then it won't be sent on to the JVM because the other debugger still needs the JVM to stay suspended.
The resume command is only sent on when both debuggers have sent it.
Otherwise, the second debugger would think the JVM is still suspended, but actually, it has been resumed by the first one.

\subfile{./graphs/suspend-proxy-2.tex}

As can be seen in figure \ref{graph:suspend-proxy-2}, the first suspend command is forwarded to the JVM, but the second one is not. 
With resume command it's the opposite: the first resume command is not forwarded, but the second one is.
It i s also important to mention that even though the command is not always forwarded to the JVM, the proxy server still has to reply to the debugger since each command packet sent from a debugger expects to receive a reply packet from the JVM.

Similar problems arise with events and event requests.
A debugger can send event requests to the JVM which will specify for which kind of events the debugger wishes to get notifications.
Later when these events occur then the JVM will send event command packets to the debugger informing what event happened and other details about the event.
These events can be about classes loaded or threads started or certain code being invoked in the application.

Events and event requests are tricky for the proxy server to route because each debugger can ask for different kind of events, but the JVM later sends the events that happened at the same time, together.
Event requests can also differ on what to do with the event when it occurs.
One debugger can specify that the whole JVM should suspend when a breakpoint is hit, but other debugger says it should just suspend a single thread.

Due to these difficulties, each event request is parsed by the proxy server and cached to later know which debugger requested which events.
When JVM sends events to the proxy server, then these events are split accordingly and each debugger receives only the events it requested.

\begin{lstlisting}[language=Java, caption={\textit{Methods for sending only the requested events to a debugger.}}, captionpos=b, label={lst:routing}]
private void sendEventsToSource(PacketSource source, CompositeEventCommand command) {
  List<VirtualMachineEvent> events = command.getEvents().stream()
      .filter(event -> isEventRequestedBySource(source, event))
      .collect(Collectors.toList());
  if (events.isEmpty()) {
    return;
  }

  proxyCommandStream.write(
      source,
      CompositeEventCommand.create(
        proxyCommandStream.getVmSource().createNewOutputId(), 
        command.getSuspendPolicy(), 
        events, vmInformation
      )
  );
}

private boolean isEventRequestedBySource(PacketSource source, VirtualMachineEvent event){
  if (event instanceof VmStartEvent || event instanceof VmDeathEvent) {
    return true;
  }
  else {
    RequestIdentifier identifier = new RequestIdentifier(
      event.getEventKind(), event.getRequestId()
    );
    List<PacketSource> sources = eventRequestIdSourceMap.get(identifier);
    return sources.contains(source);
  }
}
\end{lstlisting}

In the code listing \ref{lst:routing}, the \inlinecode{CompositeEventCommand} object has been parsed from a command packet received from the JVM.
The command is then split into a separate command for each debugger with only the events which it has requested.
The JVM start and death events are always sent to each debugger as according to the JDWP specification~\cite{oracle_jdwp_spec}.

\end{document}
