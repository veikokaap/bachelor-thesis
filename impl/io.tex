%!TEX root = ../thesis.tex
\documentclass[..thesis.tex]{subfiles}

\begin{document}

For handling the packets read from the JVM and the debuggers and the packets written to them, it was decided to use non-blocking IO with message passing.
A single thread is handling all the reads and writes regardless of the number of debuggers connected.
That thread waits until any stream from debugger or JVM is ready for writing or reading and then checks whether that stream has something to read from or write to.
If there is something to be read from the stream, then the packet bytes are read and then the packet is added to a read queue.
For writing to the streams, there is a write queue for each stream and if there is anything in that queue, then it is written to the stream.
These queues are used for avoiding concurrency problems with other threads.
Other threads don't have a direct access to the streams, but must instead get all the read packets from the read queue and if they wish to write something to a stream, then they'll add it to the write queue.

\subfile{./graphs/io.tex}

Figure~\ref{graph:io} shows how using the separate thread and queues for transport keeps the reads and writes separate from rest of the proxy server logic and guarantees that only one packet is read from or written to a stream at a time.

\end{document}
