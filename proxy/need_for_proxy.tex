%!TEX root = ../thesis.tex
\documentclass[..thesis.tex]{subfiles}

\begin{document}

This section will describe the need for a proxy server for JDWP.
More general use cases for using multiple debuggers in parallel were described in Section~\ref{sec:debugging_usecases}, 
but this section will focus in more detail on specific use cases unique for JPDA.

The JVM has support for tools called Java agents which use application programming interfaces (APIs) provided by the JVM to instrument programs running on the JVM.
These APIs give JVM tool creators extensive functionality for creating simple to use agents which can be used for solving many problems.
Agents can be used for monitoring performance, exceptions, logging or even to add previously unavailable features to the JVM, like reloading code at runtime.

However, since the instrumentation APIs provide so many possibilities, it is also quite easy to misuse the functionality and create hard to understand cases where the program doesn't function as expected.
For such cases, interactively debugging the agent and the program running at the same time is one of the few possible ways to locate the cause of the problem.
It doesn't, however, make much sense to have the running application and the agent in the same IDE workspace since their functionality and purpose are completely different.
In such case, being able to write both projects in different workspaces but to debug both of them at the same time is crucial.

Similarly, any application framework which allows the creation of plugins can benefit from being able to attach multiple debuggers.
For example, most IDEs are designed so that all the supported languages and special features are not part of the core code base, but instead available as plugins. 
Such approach has many benefits including making the main code base smaller and easier to maintain.
It also creates a clear separation between what the IDE should do and what each plugin does and makes it easier to later add new functionality by implementing a new plugin.
Since separation is so important in such cases, then it makes sense to develop all plugins in separate workspaces.
There again debugging them at once makes sense since at runtime the IDE with all the plugins functions as a single application creating a need for attaching multiple debuggers to the JVM.

\end{document}
