%!TEX root = ../thesis.tex
\documentclass[..thesis.tex]{subfiles}

\begin{document}

"Java Platform Debugger Architecture (JPDA) is a multi-tiered debugging architecture that allows tools developers to easily create debugger applications which run portably across platforms, virtual machine (VM) implementations and JDK versions."\cite{oracle_jpda_spec} \todo{Write in your own words}

Its three layers are Java Virtual Machine Tool Interface (JVM TI), Java Debug Wire Protocol (JDWP) and Java Debug Interface (JDI). 
A developer who intends to use JPDA can hook into JPDA on any of these layers. \cite{oracle_jpda_spec} \todo{Use different cites for each paragraph}

\subfile{./graphs/jpda.tex}

Since JDI is the highest level and provides the best ease of use, then Oracle encourages developers to use that for simple Java language based debuggers.\cite{oracle_jpda_spec}
While it is also possible to use the low level functionality provided by JDWP or JVM TI , they tend to require a lot more development hours depending on the task. 

\end{document}
