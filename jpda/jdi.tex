%!TEX root = ../thesis.tex
\documentclass[..thesis.tex]{subfiles}

\begin{document}

Java Debug Interface (JDI) is a Java interface which defines information and requests at a user code level.
It serves as the front-end to JDWP.
This interface is the most common way to debug a JVM and most of the Integrated Development Environments (IDEs) also use it for implementing their graphical user interface for the debugger.

It's biggest disadvantage is that the interface is available only for JVM languages.
If a developer wished to create a debugger for the JVM in some other language, then it would be necessary to implement a new debug interface using the JDWP in that particular language.
For that reason, JDI might be considered the least important layer of the JPDA since it is the easiest to exchange it for another interface and in many cases it is even mandatory to do just that.

\end{document}
