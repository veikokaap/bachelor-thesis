\documentclass{style/bachelor-thesis}
\newcommand{\articleName}{Java Virtual Machine \\ multi-debugger proxy server}
\newcommand{\articleNameEE}{Java virtuaalmasina mitme siluri puhverserver}

% shortcuts for brackets/parens
\newcommand{\lp}{\left(}
\newcommand{\rp}{\right)} 

\newcommand{\lb}{\left\lbrace}
\newcommand{\rb}{\right\rbrace}

\newcommand{\lbk}{\left[}
\newcommand{\rbk}{\right]}

\newcommand{\lllb}{\left\llbracket}

\newcommand{\rrrb}{\right\rrbracket}

% math-mode separator for inteferences
\newcommand{\separ}{\hspace{3em}}

\newcommand{\vsepar}{\vspace{2em}}

% inference macros

\newcommand{\intraproc}{\hookrightarrow}
\newcommand{\interproc}{\dashrightarrow}
\newcommand{\intrathread}{\twoheadrightarrow}
\newcommand{\interthread}{\Rrightarrow}


% other macros
\newcommand{\allstates}{\mathcal{S}}
\newcommand{\descriptor}{\Delta}
\newcommand{\absint}{*}
\newcommand{\analyze}{\mathcal{A}}
\newcommand{\cfgcode}[1]{\texttt{#1}}
\newcommand{\mcode}[1]{\text{\texttt{#1}}}
\newcommand{\figcode}[1]{\texttt{#1}}
\newcommand{\figsubcode}[2]{$\text{\figcode{#1}}_{\text{\figcode{#2}}}$}
\newcommand{\inlinecode}[1]{\texttt{#1}}
\newcommand{\inlinesubcode}[2]{$\text{\figcode{#1}}_{\text{\figcode{#2}}}$}
\newcommand{\mathword}[1]{\mathit{#1}}


% todo notes
\newcommand{\kalmer}[1]{\todo[color=red!60,inline]{Kalmer: #1}}
\newcommand{\toadd}[1]{\todo[color=red!60,inline]{#1}}

\newcommand{\toask}[1]{\todo[color=orange!40,inline]{#1}}
\newcommand{\todisc}[1]{\todo[color=orange!80,inline]{#1}}

\newcommand{\tosup}[1]{\todo[color=blue!30,inline]{#1}}
\newcommand{\toans}[1]{\todo[color=blue!60,inline]{#1}}


\newcommand{\tocomment}[1]{\todo[inline]{#1}}
\newcommand{\toguide}[1]{\todo[color=green!80,inline]{#1}}
% \newcommand{\toguide}[1]{}


\title{Java Virtual Machine multi-debugger proxy server}

%%% BEGIN DOCUMENT
\begin{document}

% BEGIN TITLE PAGE
\thispagestyle{empty}
\begin{center}

\large
UNIVERSITY OF TARTU\\[2mm]
Institute of Computer Science\\
Computer Science Curriculum\\[2mm]

%\vspace*{\stretch{5}}
\vspace{25mm}

\Large Veiko Kääp

\vspace{4mm}

\huge \articleName

%\vspace*{\stretch{7}}
\vspace{20mm}

\Large Bachelor's Thesis (9 ECTS)

\end{center}

\vspace{2mm}

\begin{flushright}
 {
 \setlength{\extrarowheight}{5pt}
 \begin{tabular}{r l} 
  \sffamily Supervisor: & \sffamily Vesal Vojdani, PhD \\
  \sffamily Supervisor: & \sffamily Märt Bakhoff, MSc
 \end{tabular}
 }
\end{flushright}

%\vspace*{\stretch{3}}
\vspace{10mm}

%{\noindent Author: .................................................................................... ``.....'' ..........\hskip16pt 2048}
\vspace{2mm}


%{\noindent Supervisor: ............................................................................... ``.....'' ..........\hskip16pt 2048}

\vspace{2mm}

%{\noindent Supervisor: ............................................................................... ``.....'' ..........\hskip16pt 2048}

\vspace{8mm}


\vfill
\centerline{Tartu 2018}

% END TITLE PAGE

% Remember to remove this from the final thesis version
\pagebreak
\listoftodos[Todos]
% END OF TODO PAGE 

% COMPULSORY INFO PAGE
\pagebreak

\selectlanguage{english}
\noindent\textbf{\large \articleName}
\vspace*{3ex}
\begin{flushleft}
  \textbf{Abstract:} %todo{Write short summary}
\end{flushleft}


\vspace*{3ex}
\begin{flushleft}
  \textbf{Keywords:} Java Virtual Machine, Debugger
\end{flushleft}
\vspace*{3ex}

\noindent\textbf{CERCS:} P170??, Computer science
\selectlanguage{estonian}

\vspace*{5ex}
\noindent\textbf{\large \articleNameEE}
\vspace*{3ex}

\begin{flushleft}
  \textbf{Lühikokkuvõte:} %todo{Write short summary} 
\end{flushleft}
\vspace*{3ex}

\begin{flushleft}
  \textbf{Võtmesõnad:} Java Virtuaalmasin, Silur 
\end{flushleft}
\vspace*{3ex}

\noindent\textbf{CERCS:} P170, Arvutiteadus

\newpage

\selectlanguage{english}



\tableofcontents

\pagebreak


%Introduction
\section{Introduction}
\subfile{introduction/introduction.tex}

\pagebreak

\section{Overview of Debugging}
\label{sec:debugging}
\subfile{debugging/debugging.tex}

\subsection{Debugging methods}
\label{sec:debugging_methods}
\subfile{debugging/methods.tex}

\subsection{Use cases of using multiple debuggers in parallel}
\label{sec:debugging_usecases}
\subfile{debugging/usecases.tex}

\pagebreak

\section{Debugging in the Java Virtual Machine}
\label{sec:jvm_debugging}
\subfile{jvm/jvm_debugging.tex} 

\subsection{Java Virtual Machine}
\label{sec:jvm}
\subfile{jvm/jvm.tex}

\subsection{Java Platform Debugger Architecture}
\label{sec:jpda}
\subfile{jpda/jpda.tex}

\subsubsection{Java Virtual Machine Tool Interface}
\label{sec:jvmti}
\subfile{jpda/jvmti.tex}

\subsubsection{Java Debug Wire Protocol}
\label{sec:jdwp}
\subfile{jpda/jdwp.tex}

\subsubsection{Java Debug Interface}
\label{sec:jdi}
\subfile{jpda/jdi.tex}

\subsection{Java Debug Wire Protocol Specification}
\label{sec:jdwp_spec}
\subfile{jpda/jdwp_spec.tex}


\pagebreak

\section{Multi-debugger connections for JVM}
\label{sec:proxy}
\subfile{proxy/proxy.tex}

\subsection{The need for connecting multiple debuggers to JVM}
\label{sec:need_for_proxy}
\subfile{proxy/need_for_proxy.tex}

\subsection{Ways to solve multi-debugger problem}
\label{sec:proxy_solutions}
\subfile{proxy/proxy_solutions.tex}

\pagebreak

\section{Implementation of JDWP proxy server}
\label{sec:implemenation}
\subfile{impl/implementation.tex}

\subsection{Connection with the JVM and debuggers}
\label{sec:impl_connection}
\subfile{impl/connection.tex}

\subsection{Sending and receiving packets}
\label{sec:impl_io}
\subfile{impl/io.tex}

\subsection{Parsing command and reply packets}
\label{sec:impl_parsing}
\subfile{impl/parsing.tex}

\subsection{Avoiding id collision}
\label{sec:impl_id}
\subfile{impl/id.tex}

\clearpage
\section{Conclusion} 

\subfile{conclusion/conclusion.tex}

\newpage

\bibliography{./bib/thesis}{}
\bibliographystyle{unsrt}

\newpage

\appendix
\section*{Appendices}
\addcontentsline{toc}{section}{Appendices}
% So that appendices would be named by letters
\renewcommand{\thesubsection}{\Alph{subsection}}
\subfile{appendix/appendix.tex}
\pagebreak
\section*{\small Non-exclusive licence to reproduce thesis and make thesis public}


I, Veiko Kääp (date of birth: 4th of January 1996),

\begin{tabbing}
\= Xiii\=\kill
\>1. \> herewith grant the University of Tartu a free permit (non-exclusive licence) to:\\\\ 

\>1.1\> 
\begin{minipage}[t]{14.2cm}
reproduce, for the purpose of preservation and making available to the public, including for addition to the DSpace digital archives until expiry of the term of validity of the copyright, and
\end{minipage}
\\\\
\>1.2 
\begin{minipage}[t]{14.2cm}
make available to the public via the web environment of the University of Tartu, including via the DSpace digital archives until expiry of the term of validity of the copyright,\\ 

\articleName\\   

supervised by Vesal Vojdani and Märt Bakhoff

\end{minipage}\\\\ 
\>2. \>I am aware of the fact that the author retains these rights.\\\\
\>3. \>
\begin{minipage}[t]{14.2cm}
I certify that granting the non-exclusive licence does not infringe the intellectual property rights or rights arising from the Personal Data Protection Act. 
\end{minipage}\\
\end{tabbing}

\noindent
Tartu, \today


\end{document}
