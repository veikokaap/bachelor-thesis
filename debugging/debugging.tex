%!TEX root = ../thesis.tex
\documentclass[..thesis.tex]{subfiles}

\begin{document}

One of the most time-taxing aspects of software development is ensuring the quality of the written program. 
That means ensuring that the program operates correctly with the given input, produces the desired output and doesn't cause any unwanted side effects.
The development of the software will begin with the description of the problem followed by construction of an algorithm designed to solve that problem. 
Then the algorithm is implemented in a programming language.

Usually, the compiler detects basic syntactic and semantic problems in the program and ensures that the program is at least able to start. 
However, the compiler won't detect if there are any logic problems in the code which can cause the output to be unexpected or the program to halt execution early with an error message.

That means the developer needs to ensure that the algorithm chosen for the problem behaves correctly in all corner cases and that the algorithm was properly translated to the programming language. 
For this, tests are written which will run the program or small isolated parts of the program and ensure that the problem terminates correctly and gives the expected output.
% * <vesal.vojdani@gmail.com> 2018-05-11T07:51:07.414Z:
% 
% > ensure that the problem terminates correctly
% Jah, kui probleemid kõik termineeruksid, siis oleks kõik korras. :)
% 
% ^.

\end{document}
