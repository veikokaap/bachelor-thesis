%!TEX root = ../thesis.tex
\documentclass[..thesis.tex]{subfiles}

\begin{document}

This thesis focuses on how to use multiple debuggers in parallel on the JVM, but that kind of functionality can be useful for any programming language. 
Most modern software design patterns preach modularity as a simple way how to manage big software projects.
So instead of having one huge project built into an application, often an application is built from many smaller projects.
Depending on the IDE a developer is using, its interactive debugger might have a limitation of only allowing the developer to debug a single project at a time.

For example, all IDEs available from the software company JetBrains have the feature of creating a separate window for each project.
And its interactive debugger can only operate on projects in the current window.
There are workaround for this problem - a developer can import external projects as a submodule to an existing project window, but that requires time to set up properly.

Another use case is when one project is developed in one IDE and the other project uses a different IDE.
Such a situation might happen due to legacy reasons - project layout and features are bind to a particular IDE - or because both IDEs have a different feature set which might benefit the projects differently.
There are also differences in functionality between different debuggers. 

Some debuggers have a completely different use case from other debuggers which makes it useful to use them in parallel.
For example, one debugger could be for monitoring the running threads and the other for looking into memory consumption.
In such a case there is no workaround for using them at once other than allowing multiple debuggers to debug an application at the same time.

\end{document}
