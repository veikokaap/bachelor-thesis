%!TEX root = ../thesis.tex
\documentclass[..thesis.tex]{subfiles}

\begin{document}

\toguide{Background to the problem.}

The Java Platform Debugger Architecture, also known as JPDA, is a widely used toolset for Java and other JVM (Java Virtual Machine) languages development.
However, one of its limitations is that there can only be a one-to-one connection between the debugger and the debugee.

\toguide{What is this thesis/paper about? What problem are we trying to solve?}

The purpose of this paper is to get rid of this limitation by creating a proxy server which would connect to the JVM and then allow multiple debuggers to connect to it. 
Such proxy server would allow the use of multiple different IDEs (Integrated Development Environment) while debugging. It would also enable to debug the JVM from multiple computers at the same time. 
No other comparable publicly known program exists at the moment.

\toguide{What are the main priorities? What do we want to achieve?}

The development of the program has two main priorities. 
First priority is that if only a single debugger is connected to the proxy server, then the behaviour of the proxy server should be identical to a situation where the debugger is connected straight to the JVM.

The second priority involves two or more debuggers connected to the server. In such case the proxy server will route and handle the sent data with the intention of creating an impression to every debugger of being connected directly to the JVM.

The main use cases when multiple debuggers are connected include setting, clearing and hitting breakpoints in code, stepping to next lines and also resuming and suspending threads.
These use cases will be covered and tested most extensively.


\end{document}
