%!TEX root = ../thesis.tex
\documentclass[..thesis.tex]{subfiles}

\begin{document}

Java Virtual Machine (JVM) is an abstract computing machine which has an instruction set and which manipulates various memory areas at run time.\cite{oracle_jvm_spec}
% * <vesal.vojdani@gmail.com> 2018-05-11T07:39:14.812Z:
% 
% See esimene lause on päris häiriv... võib-olla pigem alustada, et "The Java platform provides not only a highly performant abstract computing machine, the Java Virtual Machine (JVM), but also contains sophisticated tools for interfacing with running applications. This functionality is specified in the Java Platform Debugger Architecture (JPDA)."  Igatahes, võiks oma töö avalõigu üle natuke enne esitamist obsessiivselt redigeerida. Siiski esmamulje loeb.
% 
% ^.
One of the important parts which makes the JVM a popular platform for creating new programming languages is its built-in debugging functionality. 
This functionality comes from the Java Platform Debugger Architecture, also known as JPDA, which is a widely used toolkit for Java and other JVM languages development.

While the JPDA provides extensive tools for debugging any application running on the JVM, it has the limitations of only accepting a single connection between the debugger and the JVM. 
It's not currently possible to attach multiple debuggers to a single JVM.

The purpose of this paper is to get rid of this limitation by creating a proxy server which would connect to the JVM and then allow multiple debuggers to connect to it. 
Such proxy server would allow the use of multiple different IDEs (Integrated Development Environment) while debugging. 
It would also enable to debug the JVM from multiple remote computers at the same time. 
While there are workarounds for this limitation, none of them provides a convenient single solution for every aspect of the limitation. 

The development of the program has two main priorities. 
First priority is that if only a single debugger is connected to the proxy server, then the behaviour of the proxy server should be identical to a situation where the debugger is connected straight to the JVM.

The second priority involves two or more debuggers being connected to the server. 
In such case, the proxy server will route and handle the data with the intention of creating an impression to every debugger of being connected directly to the JVM.

The main use cases when multiple debuggers are connected include setting, clearing and hitting breakpoints in code, stepping to next lines and also resuming and suspending threads.
These use cases will be covered and tested most extensively.

The following sections will introduce the reader to the Java Virtual Machine, the Java Platform Debugger Architecture and will walk through the creation of the proxy server which will provide a solution to the limitation of connecting only a single debugger to the JVM.

\end{document}
